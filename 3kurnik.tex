 
\documentclass[russian,utf8,nocolumnxxxi,nocolumnxxxii]{eskdtext}
\usepackage[T1,T2A]{fontenc}
\usepackage[utf8]{inputenc}
\usepackage{amssymb,amsmath}
\usepackage{float}
\usepackage{tikz}
\usepackage{rotating}
\usepackage{graphicx}
\graphicspath{{pictures/}}
\DeclareGraphicsExtensions{.pdf,.png,.jpg}
\usepackage{pgfplots}
\usepackage{lipsum}
\usepackage{nccmath}
\usepackage{siunitx}
\usepackage[european,cuteinductors,smartlabels]{circuitikz}
\usepackage[backend=biber]{biblatex}

\begin{document}
\newpage
3). ЗАДАЧА ОПТИМАЛЬНОГО РАСПРЕДЕЛЕНИЯ НЕОДНОРОНЫХ РЕСУРСОВ\\
Требуется решить следующую задачу оптимального распределения неоднородных ресурсов. Пусть в распоряжении завода железобетонных изделий (ЖБИ) имеется m видов сырья (песок, щебень, цемент) в объемах ${\bf a_i}$  .Требуется произвести продукцию {\bf n} видов. Дана технологическая норма $c_ij$  требления отдельного i-го вида сырь для изготовления единицы продукции каждого j-го вида. Известна прибыль $\pi_j$  получаема от выпуска единицы продукции j-го вида. Требуется определить, какую продукцию и в каком количестве должен производить завод ЖБИ, чтобы получить максимальную прибыль.\\
Исходные данные:\\
\begin{figure}[H]
\begin{center}
\begin{minipage}[h]{0.65\linewidth}
\center{\includegraphics[width=1\linewidth]{1.png}}  \\
\end{minipage}
\end{center}
\end{figure}
Так как данная задача является целочисленной задачей линейного программирования (ILP), стандартная функция мат. пакета «SciLab» для решения задач линейного программирования karmarkar(…)не даст верного решения, если оптимальное решение для соответствующей задачи без целочисленного ограничения не является целочисленным или «близким» к нему.

Для решения задачи воспользуемся функций lp_solve из пакета lpsolve:

[x,f] = lp\_solve(c,A,b,e,vlb,[],xint), где:

A – матрица значений технологической норм

b – вектор ограничений на объем используемого сырья

c – вектор значений целевой функции - прибыли (значения вектора положительны, так как данная функция решает задачу максимизации целевой функции)

e – вектор, определяющий оператор отношения для ограничений (≤, ≥, =)

vlb – вектор, задающий нижнюю границу переменных решения

xint – вектор, задающий целочисленное ограничение на переменные

x – вектор решения, доставляющий максимум целевой функции

Листинг кода:

A = [9,5,2,9;10,8,3,5;9,9,1,8];

b = [18,15,20]’;

c = [40,60,20,25];

e = [-1,-1,-1];

vlb = [0,0,0];

xint = [1,2,3,4];

[x,f] = lp\_solve(c,A,b,e,vlb,[],xint)

x =

0.

1.

2.

0.

f =

100.

Таким образом, искомым целочисленным решением доставляющим максимум целевой функции является вектор [0;1;2;0], а значением целевой функции, отвечающему этому вектору, - 100.

Для достижения максимальной прибыли в сто условных единиц предприятию необходимо произвести одну единицу изделия №2 и две единицы изделия №3.
\newpage
{\bf 7. Выводы}

Были изучены встроенные функции математического пакета «SciLab» и операторы системы компьютерной алгебры «Reduce». Полученные знания были применены при решении задач: нахождения нулей функции, её аналитического исследования, интерполяции кубическими сплайнами функции от одной переменной, целочисленного линейного программирования.
\newpage
{\bf8. Список литературы}
\\1. Reduce. User’s manual
\\2. Introduction in SciLab
\\3. Optimization in SciLab
\\4. easyprog.ru
\\5. bsstudy.net

\end {document}



