 \documentclass[russian,utf8,nocolumnxxxi,nocolumnxxxii]{eskdtext}
\usepackage[T1,T2A]{fontenc}
\usepackage[utf8]{inputenc}
\usepackage{amssymb,amsmath}
\usepackage{float}
\usepackage{tikz}
\usepackage{rotating}
\usepackage{graphicx}
\graphicspath{{pictures/}}
\DeclareGraphicsExtensions{.pdf,.png,.jpg}
\usepackage{pgfplots}
\usepackage{lipsum}
\usepackage{nccmath}
\usepackage{siunitx}
\usepackage[european,cuteinductors,smartlabels]{circuitikz}
\usepackage[backend=biber]{biblatex}

\begin{document}

\\{\bfСодержание}
\\1. Цель и тема курсовой работы
\\2. Задание на курсовую работу
\\3. Введение
\\4. Исследование функции
\\5. Исследование кубического сплайна
\\6. Задача оптимального распределения неоднородных ресурсов
\\7. Список литературы
\newpage

 {\large\bf 1. Цель и тема курсовой работы}

\\{\bfЦель курсовой работы:} уметь применять персональный компьютер и математические пакеты прикладных программ в инженерной деятельности.
\\{\bfТема курсовой работы:} решение математических задач с использованием математического пакета «SciLab» и системы компьютерной алгебры «Reduce».

\newpage
{\large\bf2. Задания на курсовую работу}
\\1. Даны функции $f(x)=\sqrt{3}sin(x)+cos(x),g(x)=cos(2x+\frac{\pi}{3})-1$
\\а)Решить уравнение f(x)=g(x).
\\б)Исследовать функцию h(x)=f(x)-g(x) на промежутке $[0;\frac{5\pi}{6}]$
\\2. Найти коэффициенты кубического сплайна, интерполирующего данные, представленные в векторах:\\
$V_{x}=[0,1.25,2,2.625,4.25]$
$V_{y}=[2,1.925,2.4,2.7,3.64]$\\
Построить на графике функции f(x),полученную после нахождения коэффициентов кубического сплайна. \\
Представить графическое изображение результатов интерполяции исходных данных различными методами с использованием встроенных функций splin(x,y,“natural”), splin(x,y,“clamped”), splin(x,y,“not\_a\_knot”), splin(x,y, “fast”), splin(x,y,“monotone”), interp(xx,x,y,d)\\
3. Решить задачу оптимального распределения неоднородных ресурсов.
Требуется решить следующую задачу оптимального распределения неоднородных ресурсов. Пусть в распоряжении завода железобетонных изделий (ЖБИ) имеется m видов сырья (песок, щебень, цемент) в объемах ${\bf a_i}$  .Требуется произвести продукцию {\bf n} видов. Дана технологическая норма $c_ij$  требления отдельного i-го вида сырь для изготовления единицы продукции каждого j-го вида. Известна прибыль $\pi_j$  получаема от выпуска единицы продукции j-го вида. Требуется определить, какую продукцию и в каком количестве должен производить завод ЖБИ, чтобы получить максимальную прибыль.

\begin{figure}[H]
\begin{center}
\begin{minipage}[h]{0.65\linewidth}
\center{\includegraphics[width=1\linewidth]{1.png}}  \\
\end{minipage}
\end{center}
\end{figure}


\newpage


\newpage
{\bf4. Исследование функции}
\\1. Даны функции $f(x)=\sqrt{3}sin(x)+cos(x),g(x)=cos(2x+\frac{\pi}{3})-1$
\\а)Решить уравнение f(x)=g(x).
\\б)Исследовать функцию h(x)=f(x)-g(x) на промежутке $[0;\frac{5\pi}{6}]$\\
{\bfРешение уравнения.}\\
Задача а) эквивалентна следующей - требуется найти корни уравнения:\\
$h(x)=\sqrt{3}sin(x)+cos(x)-cos(2x+\frac{\pi}{3})-1$\\

{\bfОтыскание численного решения.}\\
Для отыскания численного решения воспользуемся стандартной функцией «SciLab» fsolve.\\
Очевидно, что функция h(x), являющаяся линейной комбинацией периодических функций, будет иметь период равный наименьшему общему кратному периодов этих функций, то есть $T_h=HOK(T_f,T_g)=HOK(2\pi,\pi)=2\pi.$ Таким образом, достаточно численно отыскать корни на отрезке $[0,2\pi]$ и получить периодическое решение. \\
Поскольку функция fsolve основана на методе Ньютона, требуется задать начальную точку или интервал для поиска корней. С целью отыскания начальных точек построим график функции h(x) на данном отрезке:\\
function y=h(x)\\
y=sqrt(3)*sin(x)+cos(x)-cos(2*x+\%pi/3)+1\\
endfunction\\
plot(0:0.01:2*\%pi,h)\\
Полученный график изображен на Рис.1.
\newpage

\begin{figure}[H]
\begin{center}
\begin{minipage}[h]{0.65\linewidth}
\center{\includegraphics[width=1\linewidth]{2.png}}  \\
\frametitle{ Рис 1. График функции h(x)}
\end{minipage}
\end{center}
\end{figure}

Исходя из вида графика можно предположить о наличии трех или четырех корней (в окрестности точки x = 4.2 функция предположительно может дважды переходить через ноль).
\\Используя полученное знание о поведении функции воспользуемся функцией fsolve:
\\$[x,v] = fsolve(x0,f)$, где:
\\x0 – вектор начальных значений для итеративного алгоритма отыскания нулей
\\f – функция, для которой осуществляется поиск нулей
\\x – вектор нулей функции, полученных при работе алгоритма из точек x0
\\v – вектор значений функции в точках x
\\Для проверки предположения о четырех корнях укажем две начальных точки поиска с разных сторон от локального максимума, находящегося около x= 4.2
\\Листинг кода:
\\x0 = [3, 3.9,4.5,5.5];
\\$[x,v] = fsolve(x0,h)$
\newpage
v =\\
-2.220D-16 0. 0. 7.772D-16\\
x =\\
2.6179939 4.1887902 4.1887902 5.759865


\\Анализируя полученные значения v можно заметить, что два из них не являются нулевыми, так как решения fsolve находятся с некоторой заданной степенью точности. Также функция действительно имеет только три корня на отрезке $[0,\pi]$..

Предполагая, что корни линейной комбинации таких функций, как sin и cos могут быть кратны pi, разделим решение на pi:

x/\%pi =

0.8333333 1.3333333 1.3333333 1.8333333

Полученные десятичные дроби напоминают о числах кратных 1/3. Разделим на это число:

3*x/\%pi =
\\2.5 4. 4. 5.5

\newpage
Теперь очевидно, что корни уравнения можно записать в следующей форме:
$$x_1=\frac{5}{6}*\pi+2n\pi,n\in Z$$
$$x_2=\frac{8}{6}*\pi+2n\pi,n\in Z$$
$$x_3=\frac{11}{6}*\pi+2n\pi,n\in Z$$
Таким образом, путем нехитрых манипуляций на основе численного решения было получено аналитическое.\\
В случае, если бы, описанное выше преобразование ускользнуло от нашего внимания мы получили бы следующее решение:
$$x_1=2.6179939+2n\pi,n\in Z$$
$$x_2=4.1887902+2n\pi,n\in Z$$
$$x_3=5.759865+2n\pi,n\in Z$$

Отыскание аналитического решения.

Для отыскания аналитического решения воспользуемся функцией solve 

solve(expr,var); где

expr – список из уравнений (то есть система)

var – список из переменных, относительно которых решаются уравнения expr

При попытке разрешить уравнение h(x)= 0 относительно x:\\
solve(sqrt(3)sin(x)+cos(x)-cos(2x+pi/3)-1,x);\\

То есть решение данного уравнения не было найдено.\\
Упростим данное уравнение, воспользовавшись двумя тригонометрическими тождествами: $$sin(x+y)=sin(x)cos(y)+cos(x)sin(y)$$
$$cos(2x)=1-2sin^2(x) $$
$$\sqrt{3}sin(x)+cos(x),g(x)-cos(2x+\frac{\pi}{3})+1$$ $$=2(sin(x)cos(\frac{\pi}{6})+cos(x)sin(\frac{\pi}{6}))+2sin^2(x+\frac{\pi}{6})$$ $$=2(sin(x+\frac{\pi}{6})+sin^2(x+\frac{\pi}{6})$$\\
и получим тривиальное уравнение, эквивалентное исходному
$$2(sin(x+\frac{\pi}{6})+sin^2(x+\frac{\pi}{6})=0$$
Применим к нему функцию solve:\\
solve(2sin(x+pi/6)*(1+sin(x+pi/6)));
и получим решение:
\newpage

\begin{figure}[H]
\begin{center}
\begin{minipage}[h]{0.65\linewidth}
\center{\includegraphics[width=1\linewidth]{5.png}}  \\
\frametitle{}
\end{minipage}
\end{center}
\end{figure}


где arbint (arbitrary integer) является произвольным целым числом:

$$x_1=\frac{5}{6}*\pi+2n\pi,n\in Z$$
$$x_2=-\frac{1}{6}*\pi+2n\pi,n\in Z$$
$$x_3=\frac{8}{6}*\pi+2n\pi,n\in Z$$
$$x_4=-\frac{4}{6}*\pi+2n\pi,n\in Z$$\\
Периодические решения для $x_3$и$x_4$совпадают, а периодическое решение для $x_2$ можно записать в виде:
$$x_2=\frac{11}{6}*\pi+2n\pi,n\in Z$$\\
Таким образом, применив два различных подхода, мы отыскали один и тот же набор корней нелинейной функции
\newpage
{\bfИсследование функции на заданном промежутке.}\\
Необходимо исследовать функцию h(x)  на промежутке $[0,5\frac{\pi}{6}]$.
Частично функция была исследована в предыдущем разделе. На заданном отрезке функция имеет ровно один корень в точке $x=5\frac{\pi}{6}$.\\
Проведем дальнейшее исследование функции с помощью системы «Reduce», как более располагающей к аналитическому изучению функции и её производных. Для начала определим функцию h в уже упрощенном виде в пространстве имен:\\
h(x):=2sin(x+pi/6)*(1+sin(x+pi/6));
Определим значение функции на концах отрезка с помощью оператора подстановки sub:
sub(exp,f)=g, где\\
g-результат, полученный при подстановке списка алгебраических выражений exp в функцию f.\\
Очевидно, подстановка выражения вида x = const в функцию, зависящую только от аргумента x,эквивалентна её вычислению в точке const.\\
Тогда:\\
sub(x=0,h) = 3/2\\
sub(x=5pi/6,h) = 0\\
Отыщем первую и вторую производную аналитически с помощью оператора df:\\
df(f,x,n) = dif, где\\
dif – аналитическая форма производной n-го порядка для функции f по переменной x.\\
Определим в пространстве имен производные первого и второго порядка и выведем их:\\
dh1:= df(h,x,1);\\
dh1:= df(h,x,2);\\
plot(dh1,x=(0..5pi/6));
\newpage
plot(dh2,x=(0..5pi/6));\\
Полученные графики для первой и второй производной представлены на Рис. 3 и Рис. 4 соответственно.
\begin{figure}[H]
\begin{center}
\begin{minipage}[h]{0.70\linewidth}
\center{\includegraphics[width=1\linewidth]{6.png}}  \\
\frametitle{Рис. 3. График первой производной функции h(x)}
\frametitle{}
\end{minipage}
\end{center}
\end{figure}

\begin{figure}[H]
\begin{center}
\begin{minipage}[h]{0.70\linewidth}
\center{\includegraphics[width=1\linewidth]{7.png}}  \\
\frametitle{Рис. 3. График второй производной функции h(x)}
\frametitle{}
\end{minipage}
\end{center}
\end{figure}
Отыщем аналитически нули первой и второй производной, используя оператор solve:\\
solve(dh1,x);
\newpage
\begin{figure}[H]
\begin{center}
\begin{minipage}[h]{0.70\linewidth}
\center{\includegraphics[width=1\linewidth]{8.png}}  \\
\frametitle{}
\frametitle{}
\end{minipage}
\end{center}
\end{figure}
solve(dh2,x);

\begin{figure}[H]
\begin{center}
\begin{minipage}[h]{0.70\linewidth}
\center{\includegraphics[width=1\linewidth]{9.png}}  \\
\frametitle{}
\frametitle{}
\end{minipage}
\end{center}
\end{figure}

Можно заметить, что из периодических решений для первой производной только $x=\frac{\pi}{3}$ принадлежит отрезку, на котором исследуется функция. То есть на данном отрезке первая производная имеет один ноль.\\
Аналитическая форма для второй производной уже не так проста для восприятия, поэтому воспользуемся оператором нахождения численного решения num\_solve, аналогичного функции fsolve в «SciLab»:\\
num\_solve(f,x = const) = f0, где\\

f0 – ноль функции f, найденный численно при начальной точке работы алгоритма x = const.

В качестве начальных точек работы алгоритма, основываясь на виде графика второй производной, зададим x = 0.2 и x = 2.4:

num\_solve(dh2,x = 0.2) = 0.111

num\_solve(dh2,x = 2.4) = 1.993

Основываясь на полученных результатах можно сказать, что функция:

1) Возрастает на (0,$\frac{\pi}{3}$)

2) Убывает на ($\frac{\pi}{3},5\frac{\pi}{6}$)

3) Имеет глобальный максимум в точке

4) Имеет глобальный минимум в точке $x=5\frac{\pi}{6}$

5) Точки перегиба x=0.111 и x=1.193

6) Выпукла вверх на (0,0.111)∪(1.193,$5\frac{\pi}{6}$)

7) Выпукла вниз на (0.111,1.193)
\end{document}
