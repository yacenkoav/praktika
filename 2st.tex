 
\documentclass[russian,utf8,nocolumnxxxi,nocolumnxxxii]{eskdtext}
\usepackage[T1,T2A]{fontenc}
\usepackage[utf8]{inputenc}
\usepackage{amssymb,amsmath}
\usepackage{float}
\usepackage{tikz}
\usepackage{rotating}
\usepackage{graphicx}
\graphicspath{{pictures/}}
\DeclareGraphicsExtensions{.pdf,.png,.jpg}
\usepackage{pgfplots}
\usepackage{lipsum}
\usepackage{nccmath}
\usepackage{siunitx}
\usepackage[european,cuteinductors,smartlabels]{circuitikz}
\usepackage[backend=biber]{biblatex}

\begin{document}

 \section{Задача №2}
{\bfЗадание:}\\
Найти коэффициенты кубического сплайна, интерполирующего данные, представленные в векторах ${\bf Vx  Vy }$.\\

Построить на одном графике: функцию ${\bf f(x)}$ и функцию ${\bf f1(x)}$, полученную после нахождения коэффициентов кубического сплайна.\\

Представить графическое изображение результатов интерполяции исходных данных различными методами.\\
\begin{tabular}[t]{|p{3em}|p{3em}|}
\hline
Vx&Vy\\
\hline
0 & 2.0\\
1.25 & 1.925\\
2.0 & 2.4\\
2.625 & 2.7\\
4.25 & 3.65\\
\hline
\end{tabular}
\hfill \break
\bigskip
{\bf График 5 в приложении.}\\

{\bf3.1)}Найдём коэффициенты канонического полинома определив матрицу  Вандермонда.\\
$$i=0..4$$
$$j=0..4$$
$$ VI_{ji}=Vx{_j}^i $$
$$XI_{j0}=1$$
$$
XI=
\begin{bmatrix}
1 & 0 & 0 & 0 & 0 \\
1 & 1.25 & 1.562 & 1.953 & 2.441\\
1 & 2.0 & 4.0 & 8.0 & 16.0\\
1 & 2.625 & 6.890 & 18.087 & 47.480\\
1 & 4.25 & 18.062 & 76.765 & 326.253\\

\end{bmatrix}
$$

\hfill \break
\bigskip
{\bf3.2)}Вычислим коэффициенты полинома:\\
$$a=XI^{-1}\cdot Vy$$
$$a^T=[2\quad-22.337\quad95.45\quad245.726\quad1451]$$
\hfill \break
\hfill \break
\hfill \break
\hfill \break
\hfill \break
\hfill \break
\end{document}
