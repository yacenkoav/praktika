 
\documentclass[russian,utf8,nocolumnxxxi,nocolumnxxxii]{eskdtext}
\usepackage[T1,T2A]{fontenc}
\usepackage[utf8]{inputenc}
\usepackage{amssymb,amsmath}
\usepackage{float}
\usepackage{tikz}
\usepackage{rotating}
\usepackage{graphicx}
\graphicspath{{pictures/}}
\DeclareGraphicsExtensions{.pdf,.png,.jpg}
\usepackage{pgfplots}
\usepackage{lipsum}
\usepackage{nccmath}
\usepackage{siunitx}
\usepackage[european,cuteinductors,smartlabels]{circuitikz}
\usepackage[backend=biber]{biblatex}

\begin{document}

{\bf2). ИCCЛЕДОВАНИЕ КУБИЧЕСКОГО СПЛАЙНА.}\\
Найти коэффициенты кубического сплайна, интерполирующего данные, представленные в векторах:\\
$V_x=[0,1.25,2,2.625,4.25]$
 $V_y[2,1.925,2.4,2.7,3.65]$\\
 Построить на графике функцию f(x), полученную после нахождения коэффициентов кубического сплайна. Представить графическое изображение результатов интерполяции исходных данных различными методами с использованием встроенных функций splin(x,y,“natural”), splin(x,y,“clamped”), splin(x,y,“not\_a\_knot”), splin(x,y, “fast”), splin(x,y,“monotone”), и interp(xx,x,y,d)
 Оценить погрешность интерполяции в точке x = 3.1. Вычислить значение функции в точке x = 2.1.\\
{\bf Нахождение коэффициентов кубического сплайна.}\\
Найдем коэффициенты кубическом сплайна, следующим образом: получив численное значение производных в точках с помощью функции splin, отыщем 4 (по количеству интервалов между узлами интерполяции) решения для системы линейных уравнений, в которых переменными являются коэффициенты сплайна:
$A_i*Cf_i=Y_i$\\
i=0,...,4-индекс интервала\\
$A_i=$
\begin{bmatrix}
1& x_i&x^2_i&x^3_i\\
1& x_i_+_1&x_i_+_1^2&x_i_+_1^2\\
0&1&2x_i&3x^2_i\\
0&1&2x_i_+_1&3x_i_+_1^2\\

\end{bmatrix}\\
$Y_i=(y_i,y_i_+_1,d_i,d_i_+_1)^T$\\
$x_i$-узлы интерполяции\\
$y_i$-значение функции в узлах интерполяции
$d_i$-значение производной в углах интерполяции
$Cf_i$-коэффициенты i-го кубического сплайна
\newpage
$Cf_i=A_i^-1*Y_i$

Код соответствующий решению уравнений (выбранное граничное условие – равенство третьих производных слева и справа для точек x_2 и x_3):

x = [0,1.25,2,2.625,4.25];

y = [2,1.925,2.4,2.7,3.64];

d = splin(x,y);

for i = 1:4

q = x(i);

w = x(i+1);

Cf (:,i) = [1,q,q^2,q^3;1,w^2,w^3;0,1,2*q,3*q^2;0,1,2*w,3*w^3] / [y(i);y(i+1),d(i),d(i+1)]

end\\
Cf =\\
\begin{bmatrix}
2.& 2.&0.3966184&-0.3966184\\
-1.0107514& -1.0107514&2.5841762&2.5841762\\
1.01931902&1.01931902&-0.7781536&-0.7781536\\
-0.2069672&-0.2069672&0.0926101&0.0926101\\
\end{bmatrix}\\
 Построим график интерполянта и интерполянта на линейных сплайнах итеративно:\\
for i=1:n\\
t = linspace(x(i),x(i+1));\\
plot(t,interpln([x;y],t,”black”);\\
plot(t,cfs(:,i)'*[ones(t); t; t.^ 2; t.^3]);\\
end
\newpage

\begin{figure}[H]
\begin{center}
\begin{minipage}[h]{0.70\linewidth}
\center{\includegraphics[width=1\linewidth]{10.png}}  \\
\frametitle{Рис. 3. График интерполянта, полученный из коэффициентов кубических сплайнов и интерполянта из линейных сплайнов}
\end{minipage}
\end{center}
\end{figure}
Найдем значение в точке x = 2.1, используя коэффициенты третьего кубического сплайна:\\
xp = 2.1;\\
yp = Cf(:,3)’*[1,xp, xp^2, xp^3];\\
yp =\\
2.4561559\\
Оценим погрешность интерполяции относительно линейного сплайна в точке x = 3.1:\\
xr = 3.1;\\
yl = interpln([x,y],xr);\\
yp = Cf(:,4)’*[1,xr, xr^2, xr^3];\\
Err = abs(yp-yl)\\
Err =\\
0.0795513\\
\newpage

\end{document}
